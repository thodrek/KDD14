\documentclass{sig-alternate}

\begin{document}

\conferenceinfo{WOODSTOCK}{'97 El Paso, Texas USA}
%\CopyrightYear{2007} % Allows default copyright year (20XX) to be over-ridden - IF NEED BE.
%\crdata{0-12345-67-8/90/01}  % Allows default copyright data (0-89791-88-6/97/05) to be over-ridden - IF NEED BE.
% --- End of Author Metadata ---

\title{Forecasting Rare Disease Outbreaks with Source Specific Spatio-temporal Topic Models}

\numberofauthors{6}
%\author{
%% 1st. author
%\alignauthor
%Ben Trovato\titlenote{Dr.~Trovato insisted his name be first.}\\
%       \affaddr{Institute for Clarity in Documentation}\\
%       \affaddr{1932 Wallamaloo Lane}\\
%       \affaddr{Wallamaloo, New Zealand}\\
%       \email{trovato@corporation.com}
%% 2nd. author
%\alignauthor
%G.K.M. Tobin\titlenote{The secretary disavows
%any knowledge of this author's actions.}\\
%       \affaddr{Institute for Clarity in Documentation}\\
%       \affaddr{P.O. Box 1212}\\
%       \affaddr{Dublin, Ohio 43017-6221}\\
%       \email{webmaster@marysville-ohio.com}
%% 3rd. author
%\alignauthor Lars Th{\o}rv{\"a}ld\titlenote{This author is the
%one who did all the really hard work.}\\
%       \affaddr{The Th{\o}rv{\"a}ld Group}\\
%       \affaddr{1 Th{\o}rv{\"a}ld Circle}\\
%       \affaddr{Hekla, Iceland}\\
%       \email{larst@affiliation.org}
%\and  % use '\and' if you need 'another row' of author names
%% 4th. author
%\alignauthor Lawrence P. Leipuner\\
%       \affaddr{Brookhaven Laboratories}\\
%       \affaddr{Brookhaven National Lab}\\
%       \affaddr{P.O. Box 5000}\\
%       \email{lleipuner@researchlabs.org}
%% 5th. author
%\alignauthor Sean Fogarty\\
%       \affaddr{NASA Ames Research Center}\\
%       \affaddr{Moffett Field}\\
%       \affaddr{California 94035}\\
%       \email{fogartys@amesres.org}
%% 6th. author
%\alignauthor Charles Ptexalmer\\
%       \affaddr{Palmer Research Laboratories}\\
%       \affaddr{8600 Datapoint Drive}\\
%       \affaddr{San Antonio, Texas 78229}\\
%       \email{cpalmer@prl.com}
%}

\maketitle
\begin{abstract}
Rapidly increasing volumes of news feeds from diverse data sources, such as online newspapers, Twitter and online blogs are proving to be extremely valuable resources in helping anticipate, detect, and forecast outbreaks of rare diseases. Especially, aggregating and analyzing the shared information from all available data sources collectively enables the effective monitoring of disease emergence and progression.

In this paper, we introduce a spatio-temporal topic model that captures not only the low-dimensional structure of data, but also the spatial and temporal topic trends.

focus on the problem of forecasting rare disease outbreaks and demonstrate how spatio-temporal topic models over health-related data sources can successfully be used to forecast outbreaks. More precisely, we present a novel framework that integrates topic models with one-class SVMs, so that modeling the underlying topic evolution and forecasting its prominence can be used as a surrogate for making near-term predictions of disease outbreaks. We demonstrate the effectiveness of our proposed technique using incidence data for Hantavirus in multiple countries of Latin America.

\end{abstract}

% A category with the (minimum) three required fields
\category{I.2.6}{Artificial Intelligence}{Learning}
%A category including the fourth, optional field follows...
\category{H.2.8}{Database Management}{Database Applications}[data mining]

\terms{Algorithms, Experimentation}

\keywords{Graphical Models, Data Integration, Temporal Analysis, Topic Models}

\section{Introduction}

\bibliographystyle{abbrv}
\bibliography{src_tm.bib}
\end{document}
